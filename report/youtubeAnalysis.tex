% !TEX encoding = UTF-8 Unicode

\documentclass[a4paper]{article}


\usepackage{color}
\usepackage{url}
\usepackage{amsthm}
\usepackage[T2A]{fontenc} % enable Cyrillic fonts
\usepackage[utf8]{inputenc} % make weird characters work
\usepackage{graphicx}
\usepackage{subcaption}
\usepackage{amsmath}
\usepackage{verbatim}
% \graphicspath{ {.} }

\usepackage[english,serbian]{babel}
%\usepackage[english,serbianc]{babel} %ukljuciti babel sa ovim opcijama, umesto gornjim, ukoliko se koristi cirilica

\usepackage[unicode]{hyperref}
\hypersetup{colorlinks,citecolor=green,filecolor=green,linkcolor=blue,urlcolor=blue}

\newtheorem*{tvrdjenje}{Tvrđenje}
\newtheorem*{hipoteza}{Hipoteza}
\theoremstyle{definition}
%\newtheorem{primer}{Пример}[section] %ćirilični primer
\newtheorem{primer}{Primer}[section]



\begin{document}

%\renewcommand{\abstractname}{Apstrakt} %pisace Sazetak ako se ne ukljuci ova naredba

\title{YouTube trending\\ \small{Seminarski rad u okviru kursa\\Istraživanje podataka\\ Matematički fakultet}}

\author{Nikola Dimitrijević \footnote{nikoladim95@gmail.com}\\
        Luka Živanović \footnote{mi14164@alas.matf.bg.ac.rs}\\
 }
%\date{9.~april 2015.}
\vspace*{-3cm}
    {\let\newpage\relax\maketitle}


%Molim Vas da kada budete predavali seminarski rad, imenujete datoteke tako da sadrže temu seminarskog rada, kao i imena i prezimena članova grupe (ili samo temu i prezimena, ukoliko je sa imenima predugačko). Predaja seminarskih radova biće isključivo preko web forme, a NE slanjem mejla.}

\tableofcontents

\newpage



% ==============================================================================
\section{Uvod}
\label{sec:uvod}
% ==============================================================================

Ukratko sta radimo.

% ==============================================================================
\section{Analiza i pretprocesiranje podataka}
\label{sec:analiza}
% ==============================================================================


% ==============================================================================
\section{Klasifikacija}
\label{sec:klasifikacija}
% ==============================================================================
....
% ==============================================================================
\section{Pravila pridruživanja}
\label{sec:pravila}
% ==============================================================================

x <= {y, z}

% ==============================================================================
\section{One vizualizacija mozda ovde?}
\label{sec:dod}
% ==============================================================================

picz
\begin{comment}

\newlength{\figWidth}
\setlength{\figWidth}{0.45\textwidth}
%accuracy
\begin{figure}[h!]
\centering
\begin{subfigure}[t]{\figWidth}
    \includegraphics[width=1\textwidth]{accuracy_a.png}
    \caption{Sa svim atributima}

\end{subfigure}
~
\begin{subfigure}[t]{\figWidth}
    \includegraphics[width=1\textwidth]{accuracy_b.png}
    \caption{Sa redukovanim skupom atributa}

\end{subfigure}

\caption{Preciznost klasifikacije u zavisnosti od algoritma}

\label{fig:acc}
\end{figure}

% false negative
\begin{figure}[h!]
\centering
\begin{subfigure}[t]{\figWidth}
    \includegraphics[width=1\textwidth]{false_pos_a.png}
    \caption{Sa svim atributima}

\end{subfigure}
~
\begin{subfigure}[t]{\figWidth}
    \includegraphics[width=1\textwidth]{false_pos_b.png}
    \caption{Sa redukovanim skupom atributa}

\end{subfigure}

\caption{Udeo lažno pozitivnih instanci u zavisnosti od algoritma}

\label{fig:falsePos}
\end{figure}

% false positive
\begin{figure}[h!]
\centering
\begin{subfigure}[t]{\figWidth}
    \includegraphics[width=1\textwidth]{false_neg_a.png}
    \caption{Sa svim atributima}

\end{subfigure}
~
\begin{subfigure}[t]{\figWidth}
    \includegraphics[width=1\textwidth]{false_neg_b.png}
    \caption{Sa redukovanim skupom atributa}

\end{subfigure}

\caption{Udeo lažno negativnih instanci u zavisnosti od algoritma}

\label{fig:falseNeg}
\end{figure}
\end{comment}


\end{document}
